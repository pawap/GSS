\documentclass[ngerman]{fbi-aufgabenblatt}

% Folgende Angaben bitte anpassen

\renewcommand{\Vorlesung}{GSS}
\renewcommand{\Semester}{SoSe 2018}

\renewcommand{\Aufgabenblatt}{1}
\renewcommand{\Teilnehmer}{Knudsen, Rasch, Runge, Titov}

\usepackage{url}

\begin{document}

%
% Aufg. 1
%
\aufgabe{Allgemeine Aussagen zur IT-Sicherheit}

% 1.1
\textbf{1.1 Verteilte Systeme (optional)}

\dq Ein Verteiltes System (VS) ist eine durch ein Kommunikationssystem lose gekoppelte Menge von Knoten, wobei

\begin{itemize} 
\item die Knoten kooperieren, um Systemfunktionen auszuführen (verteilte systemweite Kontrolle)
\item keine zwei Prozesse dieselbe Sicht des Systemzustands besitzen und insbesondere kein zentraler Prozess existiert, 
     der andere Prozesse mit einer konsistenten, identischen Sicht des globalen Systemzustands versorgen kann.\dq
\end{itemize}

(Quelle: Prof. Dr. rer. nat. Bernd E. Wolfinger - Datenkommunikation und Rechnernetze (DKR) Skript, S. 15)

\dq Die 3 wichtigsten Aspekte eines verteilten Systems sind:

\begin{itemize} 
\item lose gekoppelte Knoten (kein gemeinsamer Speicher; Kommunikation nur durch Nachrichten)
\item verteilte systemweite Kontrolle
\item kein globaler Systemzustand\dq
\end{itemize}

(Quelle: Prof. Dr. rer. nat. Bernd E. Wolfinger - Datenkommunikation und Rechnernetze (DKR) Foliensatz 1, Folie 44)

3 Beispiele für verteilte Systeme:
\begin{itemize} 
\item Terminalnetze (bestehend aus Terminals und Terminalkonzentratoren)
\item Mobile Systeme (bestehend z.B. aus Endgerät und Peripherie-Komponenten wie Kamera, Drucker, Wifi-Festplatte,...)
\item ein beliebiger Abschnitt des Internets, der aus mehr als einem Rechner besteht
\end{itemize}

% 1.2
\textbf{1.2 Sicherheit verteilter Systeme (optional)}

Vorteile: 
\begin{itemize}
\item Wenn ein Verteiltes System mit Verfügbarkeitsverbund vorliegt, kann nach einem erfolgreichem Angriff auf einen Knoten die Funktionalität des Gesamtsystems weiterhin gewährleistet werden. 
\item Wenn ein Verteiltes System mit Datenverbund vorliegt, kann nach einem erfolgreichem Angriff auf einen Knoten mit einhergehendem Datenverlust die Verfügbarkeit der Daten durch einen  anderen Knoten weiterhin gewährleistet werden.
\item ...
\end{itemize}

Nachteile: 
\begin{itemize}
\item Durch Infektion eines Rechners können die anderen Rechner des verteilten Systems leichter infiziert werden, u.a. da zwischen ihnen regelmäßig Nachrichten ausgetauscht werden
\item Umfang der Sicherheitsvorkehrungen und Einfachheit des Zugriffs stehen in Konkurrenz zu einander
\item ...
\end{itemize}

% 1.3
\textbf{1.3 Ursachen (Pflicht, 6 Punkte)}

Vermutungen:

\begin{itemize} 
\item Kosten
\item Unwissen
\item Fahrlässigkeit
\end{itemize}

Tatsächlich eher: 

\begin{itemize} 
\item Defizite in der Zusammenarbeit von IT und Sicherheitsverantwortlichen
\item Mangelnde oder mangelhafte Sicherheitskonzepte für neue Technologien (z.B. Mobile Endgeräte, Cloud, ...)
\item Mangel an Collaboration und Ressourcen
\end{itemize}

\begin{itemize} 
\item immer mehr verwundbare Endpunkte, 
\item unwirksame Strategien in der technologischen Implementierung und organisatorischen Priorisierung von IT-Sicherheit 
\item sowie die Unfähigkeit, Mitarbeitern Best Practices nahezubringen.
\end{itemize}

(Quelle: \url{https://www.ponemon.org/local/upload/file/Third_Annual_Study_Patient_Privacy_FINAL.pdf})

Referenzen: 
Studien, z.B. von <kes> – Die Zeitschrift für Informations-Sicherheit, die Microsoft-Sicherheitsstudien, 
oder auch die oben verlinkte Studie vom Ponemon Institute.

Komplikationen: 
\begin{itemize}
\item Unternehmen werden unter Umständen nicht dran interessiert sein,
Probleme ihres Umgangs mit IT-Sicherheit zu offenbaren, zumal der Punkt \dq Unwissen \dq
die kritische Auseinandersetzung damit erschwert bzw. evtl. sogar unmöglich macht.
\item Studien werden evtl. durch die Auftraggeber oder die Durchführenden selbst insofern kompromittiert, 
als dass die dargestellten Ergebnisse nicht (ganz) der Realität entsprechen.
\item Befragte haben schlichtweg keine Fachkentnisse und bieten somit ein verzerrtes Bild der Realität.
\end{itemize}

% 1.4
\textbf{1.4 Digitale Signaturen (optional)}

Im Gegensatz zu einer Signatur im Sinne einer physischen Unterschrift mit Stift auf Papier handelt es sich
\dq bei der digitalen Signatur (DSig) [...] um einen asymmetrischen elektronischen Schlüssel, 
der die Identität des Benutzers sicherstellt. Der Schlüssel wird mit dem privaten Schlüssel des Absenders verschlüsselt 
und vom Empfänger mit dem öffentlichen Schlüssel gelesen.\dq

(Quelle: \url{https://www.itwissen.info/Digitale-Signatur-digital-signature-DSig.html})


\newpage
%
% Aufg. 2
%
\aufgabe{Schutzziele}

% 2.1
\textbf{2.1 Abgrenzung I (Pflicht, 14 Punkte)}

a) Anonymität, Pseudonymität und Unbeobachtbarkeit

Während die Identität eines Akteur bei gewahrter \emph{Anonymität} lediglich nicht preisgegeben wird,\\
wird bei \emph{Pseudonymität} eine falsche Identität vorgeschoben, wobei die wahre Identität bei Eintritt besonderer Rahmenbedingungen ermittelt werden kann,\\
und bei \emph{Unbeobachtbarkeit} ist weder seine wahre, noch eine vorgetäuschte Identität ersichtlich. Seine Handlungen können nicht beobachtet werden oder es kann nicht festgestellt werden, dass die Handlungen einem einzelnen Akteur zuzuordnen sind.  

b) Vertraulichkeit und Verdecktheit

Während bei \emph{Vertraulichkeit} die Daten eines Akteurs vor Blicken Dritter geschützt sind,

ist bei \emph{Verdecktheit} das Wissen über das Stattfinden einer Übertragung an sich vor Dritten geschützt.

c) Integrität und Zurechenbarkeit

Bei \emph{Integrität} wird versucht, eine Veränderung der Daten zu verhindern, 

während bei \emph{Zurechenbarkeit} die Veränderung der Daten einem Akteur nachgewiesen werden kann.

% 2.2
\textbf{2.2 Techniken (optional)}


\newpage
\aufgabe{Angreifermodell}
% Aufg. 3

% 3.1
\textbf{3.1 Angreifermodell (optional)}

Das Angreifermodell definiert die maximal berücksichtigte Stärke eines
Angreifers, gegen den ein Schutzmechanismus gerade noch wirkt.

Es beschreibt:
\begin{itemize}
\item Rollen des Angreifers (Außenstehender, Benutzer, Betreiber,
Wartungsdienst, Produzent, Entwerfer …), auch kombiniert
\item Verbreitung des Angreifers (Stellen im System, an denen der
Angreifer Informationen gewinnen oder Systemzustände verändern kann)\\
\item Verhalten des Angreifers
	\begin{itemize}
   \item passiv / aktiv, beobachtend / verändernd
   \end{itemize}
\item Rechenkapazität des Angreifers
	\begin{itemize}
   \item unbeschränkt: informationstheoretisch
   \item beschränkt: komplexitätstheoretisch
   \end{itemize}
\end{itemize}

(Quelle: Prof. Dr. Hannes Federrath - Sicherheit in verteilten Systemen (SVS)-Foliensatz 1 - \dq Einführung in die IT-Sicherheit\dq , Folie 26)

% 3.2
\textbf{3.2 Praxisbeispiel (Pflicht, 10 Punkte)}

Angreifermodell für das Abheben von
Bargeld an Geldautomaten mit einer EC-Karte

%% PAW %%
%% Frage: Sollen alle möglichen Szenarien in einem Modell abgedeckt werden?

Angreifermodell 1

Rolle: Außenstehender, weiterer Benutzer (aber auch andere Insider) \\
Verbreitung: Sichtkontakt zum Automaten (PIN-Eingabe)/ ggf. über Kamera/Spiegel \\
	oder Anbringen einer Vorrichtung zur Ermittlung der PIN 
Verhalten: passiv, beobachtend / aktiv, beobachtend (Sicherheitsdienst)  / aktiv, verändernd (Spiegel, Vorrichtung, unerlaubte Kameranutzung)\\
Rechenkapazitäten: unbeschränkt. \\

Angreifermodell 2

Rolle: Außenstehender, weiterer Benutzer (aber auch andere Insider) \\
Verbreitung: Besitz einer betrügerisch erworbenen gültigen EC-Karte oder Erlangen der EC-Kartendaten durch unerlaubte Handlungen \\
Verhalten: aktiv, verändernd \\
Rechenkapazitäten: beschränkt.   

%
% Aufg. 4
%
\newpage
\aufgabe{Angriffsformen}

% 4.1
\textbf{4.1 Essenslieferungen (Pflicht, 7 Punkte)}

% 4.2
\textbf{4.2 Liste bekannter WLAN-AP SSIDs (optional)}


\newpage
\aufgabe{Passwortsicherheit}
% Aufg. 5
%Ich (Jonas) arbeite ab morgen (12.4.) an Aufgabe 5, speziell 5.6 

% 5.1
\textbf{5.1. Eigenschaften kryptographischer Hashfunktionen (optional)}

% 5.2
\textbf{5.2. Einfaches Hash-Verfahren (optional)}

% 5.3
\textbf{5.3. Brute-Force-Angriff (Pflicht, 6 Punkte)}

% 5.4
\textbf{5.4. Time-Memory-Trade-Off (optional)}

% 5.5
\textbf{5.5. Salting (optional)}

% 5.6
\textbf{5.6. Dictionary-Attack (Pflicht, 10 Punkte)}
\begin{verbatim}
#!/bin/bash
## user;salt:hash
#berta;xohth4dew5p8:14146888a9cb5e924987691876fb4252
#PW="1234"
salt="xohth4dew5p8"
#hash=$(printf '%s' $PW$salt | md5sum | cut -d ' ' -f 1)
hash=14146888a9cb5e924987691876fb4252
#numofcpu=4
numofcpu=$(cat /proc/cpuinfo | grep processor | wc -l)

for ((z=1;z<=$numofcpu;z++)); do
 for word in $(split --number=l/$z/$numofcpu ./deutsch.txt); do 
  length=$(echo $word | wc -m)
    if [ $length -lt 8 ] ; then
      word=$(echo $word | sed 's/\([A-Z]\)/\L\1/g')
      #echo $PW$word | md5sum
      #echo $word
      var=$(printf '%s' $salt$word | md5sum | cut -d ' ' -f 1)
      #echo $var
      if [ $var = $hash ] ; then
        echo -e "Passwort gefunden: $word"
      fi
    fi
  done & echo $z
done
Passwort: sterne
\end{verbatim}

\end{document}
